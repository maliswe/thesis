\section{Methodology}
This study employs a mixed-methods approach to analyze the trade-offs between productivity and security in AI-assisted software development.

\subsection{Research Method}
A combination of surveys and code analysis will be used. Developer surveys have been widely used to assess AI tool adoption and perceived productivity benefits \cite{developer-adoption, snyk2023}. Similarly, security analysis of AI-generated code has been employed in prior studies to assess vulnerability rates in AI-assisted programming \cite{fu2023, perry2023, asare2024}.

\subsection{Data Collection}
\begin{itemize}
    \item \textbf{Developer Survey:} An online survey with 40-50 software developers focusing on AI tool usage, productivity impacts, and security awareness. The survey structure is inspired by prior studies on developer adoption of AI tools \cite{developer-adoption, snyk2023}.
    \item \textbf{Code Analysis:} Examination of 25-30 AI-generated code samples for security vulnerabilities, following methodologies used in existing research on Copilot-generated code security risks \cite{fu2023, asare2024}.
\end{itemize}

\subsection{Data Analysis}
Survey responses will be analyzed using statistical methods to identify patterns in productivity perceptions and security concerns. Code analysis will involve security testing using automated vulnerability detection tools, similar to approaches used in studies evaluating AI-generated security flaws \cite{fu2023, perry2023}.

\subsection{Threats to Validity}
\begin{itemize}
    \item \textbf{Internal Validity:} Potential biases in self-reported survey responses.
    \item \textbf{External Validity:} The extent to which findings generalize to diverse development environments.
    \item \textbf{Construct Validity:} Ensuring that productivity and security are measured using established frameworks from prior research \cite{developer-adoption, fu2023}.
\end{itemize}
