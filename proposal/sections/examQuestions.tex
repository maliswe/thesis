\section*{Exam Questions}

\subsection*{Q1: You are supposed to explore the uses of generative AI in software engineering education via a survey. Write 5 survey questions that would get the core of that content and explain why they cover the research question well. (5 questions plus about 200 words of explanation.)}

Survey Question:

Q1: How often do you use Generative AI (For example, ChatGPT, Gemini ….) in your studying as a software engineering student/lecturer?

Explanation:
The first question focuses on identifying the groups using Generative AI tools in software engineering (students, teachers, or professionals). We want to understand whether these tools are becoming part of the education process and how often they are used in studying or solving daily tasks. Frequent use can indicate that users find them helpful, potentially filling gaps in current educational practices. We also aim to analyze the acceptance and trust in AI tools over time and determine whether educational institutions are open to these technologies or still cautious.

Q2: What type of tasks do you usually use Generative AI for in software engineering education? (For example: coding, debugging, writing reports, gathering requirements ….)

Explanation:
This question explores which tasks are most commonly supported by Generative AI. It helps identify areas where users feel the most need for assistance, such as when stuck in development or needing help with ideation. It reveals whether AI acts more as a learning assistant, a coding partner, or something else.

Q3: When you use generative AI, do you feel like it actually helps you learn or improves your understanding? Why or why not?

Explanation:
This question assesses whether AI tools contribute to actual learning or are simply used for completing tasks. It aims to uncover whether users are thinking deeply or becoming overly dependent. The responses will help evaluate if AI is enhancing problem-solving abilities or hindering genuine understanding.

Q4: While doing an assignment or project, what is the instructor's opinion of using AI tools such as ChatGPT, Gemini, etc.?

Explanation:
This question investigates educators' perspectives on Generative AI. Are they supportive or skeptical? Understanding the institution’s attitude can help determine if AI use is accepted or discouraged. It also exposes any discrepancies between student and instructor viewpoints.

Q5: What concerns do you have when you use Generative AI in your Software Engineering learning?

Explanation:
This question uncovers potential risks and concerns such as dependence, cheating, lack of understanding, or buggy AI-generated code. It gives a more comprehensive view of user perspectives and helps balance the benefits and drawbacks of AI in software engineering education.

\subsection*{Q2: When designing surveys, which different types of sampling exist? Name at least three types of sampling and give one concrete example for each (not probabilistic and non-probabilistic is too high-level, be more specific).}

Choosing the right sampling method when designing surveys depends on the aim of the research, the goal, the size of the population, and the available time and access. Below are three specific types of sampling:

\textbf{Simple Random Sampling:} This is used to get a fair representation of a large population.
\newline
\textit{Example:} If we have access to a student list across departments, we randomly pick 100 students to survey about their use of Generative AI. Each student has an equal chance, which makes results more generalizable.

\textbf{Stratified Sampling:} Used when comparing specific subgroups.
\newline
\textit{Example:} To compare AI use between Computer Science and Science students, we divide them into those two groups and sample each separately to avoid overrepresenting one group.

\textbf{Convenience Sampling:} Useful when there is limited time or access.
\newline
\textit{Example:} In a bachelor thesis, we ask our classmates to fill a survey on efficiency vs. security in AI tools. It’s quick but may lack broader accuracy.

\subsection*{Q3: Imagine an experimental setup evaluating the effects of developer experience on program quality.}

In this experiment, the aim is to measure how developer experience influences software quality.

\textbf{Independent Variable:} Developer experience level.

\textbf{Levels:}
\begin{itemize}
  \item Low experience (<2 years full time)
  \item Medium experience (2--5 years)
  \item High experience (>5 years)
\end{itemize}

\textbf{Dependent Variable:} Program quality, measured by bug count, performance, code readability, or alignment with requirements.

\textbf{Objects:} Programming tasks assigned to all participants.

\textbf{Subjects:} Developers performing the tasks.

\textbf{Control Variables:}
\begin{itemize}
  \item Same programming language/framework
  \item Same task complexity
  \item Same time allocation
  \item Same development environment
\end{itemize}

\subsection*{Q4: As part of your planned thesis work, what sort of ethical issues may arise and why? What are you planning to do to mitigate ethical issues?}

Our thesis, “Trade-offs between efficiency and security in AI-assisted software development,” may encounter several ethical issues. The main one is collecting data from real developers, which involves gathering sensitive information, such as personal opinions on tools and security practices.

To mitigate this, we will ensure:
\begin{itemize}
  \item Clear informed consent from participants.
  \item Anonymous and confidential handling of responses.
  \item Secure storage of all data.
  \item Transparent information about how the data will be used.
\end{itemize}

This protects participant privacy and meets ethical standards.

\subsection*{Q5: Design Science and Action Research have several similarities, but also some key differences.}

\textbf{Similarities:}
\begin{itemize}
  \item Both aim to solve practical, real-world problems.
  \item Both are iterative and involve evaluation and improvement.
  \item Both engage users or stakeholders in the process.
\end{itemize}

\textbf{Differences:}
\begin{itemize}
  \item \textbf{Design Science (DS):} Focuses on creating and evaluating artifacts like models, systems, or tools. Its goal is to contribute to knowledge by designing useful innovations.
  \item \textbf{Action Research (AR):} Aims to solve issues within a specific organization or context. It requires active participation from researchers and stakeholders and emphasizes reflection and change within the setting.
\end{itemize}

DS is more about building and proving the utility of something new, while AR is about improving existing situations through collaborative cycles.

